\documentclass[a4paper]{book}
\usepackage[times,inconsolata,hyper]{Rd}
\usepackage{makeidx}
\usepackage[utf8,latin1]{inputenc}
% \usepackage{graphicx} % @USE GRAPHICX@
\makeindex{}
\begin{document}
\chapter*{}
\begin{center}
{\textbf{\huge Test of foo}}
\par\bigskip{\large \today}
\end{center}
\inputencoding{utf8}
\HeaderA{MSOpt}{MSOpt}{MSOpt}
%
\begin{Description}\relax
The \code{MSOpt} function allows the user to define the
structure of the experiment, the set of optimization criteria and the a priori
model to be considered. The output is a list containing all information about
the settings of the experiment. According to the declared criteria, the list
also contains the basic matrices for their implementation, such as
information matrix, matrix of moments and matrix of weights. This function
returns the \code{msopt} argument of the \code{\LinkA{Score}{Score}} and
\code{\LinkA{MSSearch}{MSSearch}} functions of the \code{multiDoE} package.
\end{Description}
%
\begin{Usage}
\begin{verbatim}
MSOpt(facts, units, levels, etas, criteria, model)
\end{verbatim}
\end{Usage}
%
\begin{Arguments}
\begin{ldescription}
\item[\code{facts}] A list of vectors representing the distribution of factors
across strata. Each item in the list represents a stratum and the first item
is the highest stratum of the multi-stratum structure of the experiment.
Within the vectors, experimental factors are indicated by progressive integer
from 1 (the first factor of the highest stratum) to the total number of
experimental factors (the last factor of the lowest stratum). Blocking
factors are differently denoted by empty vectors.

\item[\code{units}] A list whose \eqn{i}{}-th element, \eqn{n_i}{}, is the number of
experimental units within each unit at the previous stratum (\eqn{i-1}{}). The
first item in the list, \eqn{n_1}{}, represents the number of experimental
units in the stratum \eqn{0}{}. The latter is defined as the entire experiment,
such that \eqn{n_0 = 1}{}.

\item[\code{levels}] A vector containing the number of available levels for each
experimental factor in \code{facts} (blocking factors are excluded). If all
experimental factors share the number of levels one integer is sufficient.

\item[\code{etas}] A list specifying the ratios of error variance between subsequent
strata. It follows that \code{length(etas)} must be equal to
\code{length(facts)-1}.

\item[\code{criteria}] A list specifying the criteria to be optimized. It can
contain any combination of:
\begin{itemize}

\item{} "I" : I-optimality
\item{} "Id" : Id-optimality
\item{} "D" : D-optimality
\item{} "A" : Ds-optimality
\item{} "Ds" : A-optimality
\item{} "As" : As-optimality

\end{itemize}

These criteria are well explained in Borrotti, Sambo, Mylona and Gilmour (2017).
More detailed information on the available criteria is also given under
\strong{Details}.

\item[\code{model}] A string which indicates the type of model, among ``main",
``interaction" and ``quadratic".
\end{ldescription}
\end{Arguments}
%
\begin{Details}\relax
A little notation is introduced to show the criteria that can be
used in the multi-objective approach of the \code{multiDoE} package. \\{}

For an experiment with \eqn{N}{} runs and \eqn{s}{} strata, with stratum \eqn{i}{}
having \eqn{n_i}{} units within each unit at stratum (\eqn{i-1}{}) and
stratum 0 being defined as the entire experiment (\eqn{n_0 = 1}{}), the
general form of the model can be written as:

\deqn{y = X\beta + \sum\limits_{i = 1}^{s} Z_i\varepsilon_i}{}

where
\eqn{y}{} is an \eqn{N}{}-dimensional vector of responses (\eqn{N = \prod_{j = 1}^{s}n_j}{}),
\eqn{X}{} is an \eqn{N}{} by \eqn{p}{} model matrix,
\eqn{\beta}{} is a \eqn{p}{}-dimensional vector containing the \eqn{p}{} fixed model parameters,
\eqn{Z_i}{} is an \eqn{N}{} by \eqn{b_i}{} indicator matrix of \eqn{0}{} and
\eqn{1}{} for the units in stratum \eqn{i}{} (i.e. the (\eqn{k,l}{})th element
of \eqn{Z_i}{} is \eqn{1}{} if the \eqn{k}{}th run belongs to the \eqn{l}{}th
block in stratum \eqn{i}{} and \eqn{0}{} otherwise) and
\eqn{b_i = \prod_{j = 1}^{i}n_j}{}.
Finally, the vector \eqn{\varepsilon_i \sim N(0,\sigma_i^2I_{b_i})}{} is a \eqn{b_i}{}-dimensional vector
containing the random effects, which are all uncorrelated. The variance
components \eqn{\sigma^{2}_{i} (i = 1, \dots, s)}{} have to be estimated and this is usually done using
the REML (\emph{REstricted Maximum Likelihood}) method.

The best linear unbiased estimator for the parameter vector \eqn{\beta}{} is
the generalized least square estimator:
\deqn{ \hat{\beta}_{GLS} = (X'V^{-1}X)^{-1}X'V^{-1}y}{}

This estimator has variance-covariance matrix:
\deqn{Var(\hat{\beta}_{\emph{GLS}}) = \sigma^{2}(X'V^{-1}X)^{-1}}{}

where
\eqn{V = \sum\limits_{i = 1}^{s}\eta_i Z_i'Zi}{},
\eqn{\eta_i = \frac{\sigma_i^{2}}{\sigma^{2}}}{} and
\eqn{\sigma^{2} = \sigma^{2}_{s}}{}.

Let \eqn{M = (X' V^{-1} X)}{} be the information matrix of \eqn{\hat{\beta}}{}.

\begin{itemize}

\item{}  \strong{\emph{D}-optimality.} It is based on minimizing the generalized
variance of the parameter estimates. This can be done either by minimizing the
determinant of the variance-covariance matrix of the factor effects' estimates
or by maximizing the determinant of \eqn{M}{}. \\{}
The objective function to be minimized is:
\deqn{f_{D}(d; \eta) = \left(\frac{1}{\det(M)}\right)^{1/p}}{}
where \eqn{d}{} is the design with information matrix \eqn{M}{} and \eqn{p}{} is the
number of model parameters.

\item{}  \strong{\emph{A}-optimality.} This criterion is based on
minimizing the average variance of the estimates of the regression coefficients.
The sum of the variances of the parameter estimates (elements of
\eqn{\hat{\beta}}{}) is taken as a measure, which is equivalent to
considering the trace of \eqn{M^{-1}}{}. \\{}
The objective function to be minimized is:
\deqn{f_{A}(d; \eta) = \texttt{tr}(M^{-1})}{}
where \eqn{d}{} is the design with information matrix \eqn{M}{}.

\item{}  \strong{\emph{I}-optimality.} It seeks to minimize the average
prediction variance. \\{}
The objective function to be minimized is:
\deqn{f_{I}(d; \eta) = \frac{\int_{\chi} f'(x)(M)^{-1}f(x)\,dx }{\int_{\chi} \,dx}}{}
where \eqn{d}{} is the design with information matrix \eqn{M}{} and \eqn{\chi}{}
represents the design region. \\{}
It can be proved that when there are \eqn{k}{} treatment factors each with two
levels, so that the experimental region is of the form \eqn{[-1, +1]^{k}}{},
the objective function can also be written as:
\deqn{f_{I}(d; \eta) = \text{trace} \left[(M)^{-1} B\right]}{}
where \eqn{d}{} is the design with information matrix \eqn{M}{} and
\eqn{B = 2^{-k} \int_{\chi}f'(x)f(x) \,dx }{} is the moments matrix.
To know the implemented expression for calculating the moments matrix for a
cuboidal design region see Hardin and Sloane (1991).

\item{}  \strong{\emph{Ds}-optimality.} Its aim is to minimize the generalized
variance of the parameter estimates by excluding the intercept from the set
of parameters of interest. Let \eqn{\beta_i}{} be the model parameter
vector of dimension (\eqn{p_i - 1}{}) to be estimated in stratum \eqn{i}{}.
Let \eqn{X_i}{} be the associated model matrix \eqn{m_i}{} by \eqn{(p_i-1)}{}, where \eqn{m_i}{} is the number of units in stratum \eqn{i}{}.
The partition of interest of the matrix of variances and covariances of
\eqn{\hat{\beta}_i}{} is
\deqn{(M_i^{-1})_{22} = [X'_i (I - \frac{1}{m_i} 11^{'})X_i]^{-1}}{} \\{}
The objective function to be minimized is:
\deqn{f_{D_s}(d; \eta) = |(M_i^{-1})_{22}|}{}

\item{} \strong{\emph{As}-optimality.} This criterion is based on minimizing
the average variance of the estimates of the regression coefficients excluding
the intercept from the set of parameters of interest. \\{}
With reference to the notation introduced for the previous criterion, the
objective function to be minimized is:
\deqn{f_{A_s}(d; \eta) = \texttt{tr}(W_i(M_i^{-1})_{22})}{}
where \eqn{W_i}{} is a diagonal matrix of weights, with the weights scaled so that
the trace of \eqn{W_i}{} is equal to 1.

\item{}  \strong{\emph{Id}-optimality.} It seeks to minimize the average
prediction variance excluding the intercept from the set of parameters of
interest. \\{}
The objective function to be minimized is the same as the
\emph{I}-optimality criterion where the first row and columns of the \eqn{B}{}
matrix are deleted.

\end{itemize}

\end{Details}
%
\begin{Value}
\code{MSOpt} returns a list containing the following components:
\begin{itemize}

\item{} \code{facts}: The argument \code{facts}.
\item{} \code{nfacts}: An integer. The number of experimental factors (blocking factors are excluded from the count).
\item{} \code{nstrat}: An integer. The number of strata.
\item{} \code{units}: The argument \code{units}.
\item{} \code{runs}: An integer. The number of runs.
\item{} \code{etas}: The argument \code{etas}.
\item{} \code{avlev}: A list showing the available levels for each experimental factor.
\item{} \code{levs}: A vector showing the number of available levels for each experimental factor.
\item{} \code{Vinv}: The inverse of the variance-covariance matrix of the responses.
\item{} \code{model}: The argument \code{model}.
\item{} \code{crit}: The argument \code{criteria}.
\item{} \code{ncrit}: An integer. The number of criteria.
\item{} \code{M}: The moment matrix. Only with \emph{I}-optimality criteria.
\item{} \code{M0}: The moment matrix. Only with \emph{Id}-optimality criteria.
\item{} \code{W}: The diagonal matrix of weights. Only with \emph{As}-optimality criteria.
This matrix assigns to each main effect and each interaction effect an absolute
weight equal to 1, while to the quadratic effects it assigns an absolute weight
equal to 1/4.

\end{itemize}

\end{Value}
%
\begin{References}\relax
R. H. Hardin and N. J. A. Sloane. Computer generated minimal (and larger)
response-surface designs: (II) The cube. Technical report, 1991.

M. Borrotti and F. Sambo and K. Mylona and S. Gilmour. A multi-objective
coordinate-exchange two-phase local search algorithm for multi-stratum
experiments. Statistics \& Computing, 2017.
\end{References}
\inputencoding{utf8}
\HeaderA{MSSearch}{MSSearch}{MSSearch}
%
\begin{Description}\relax
The \code{MSSearch} function can be used to obtain an optimal
multi-stratum experimental design considering one or more optimization criteria,
up to a maximum of six criteria simultaneously. \\{}
This function implements the procedure MS-Opt proposed by Sambo, Borrotti,
Mylona e Gilmour (2017) as an extension of the Coordinate-Exchange Algorithm
for constructing exact optimal experimental designs. This innovative procedure,
instead of minimizing a single objective function as in the CE algorithm,
seeks to minimize the following scalarization of the objective functions for
all criteria under consideration:
\deqn{f_W = \sum_{c \in C}{\alpha_cf_c(d; \eta)=\overline{\alpha} \cdot \overline{f}},}{}
with
\deqn{\sum_{c\inC}(\alpha_c) = 1}{}
where \eqn{c}{} is the set of criteria to be minimized, \eqn{f_c}{} is the objective function
for the \eqn{c}{} criterion and \eqn{\overline{\alpha}}{} is the vector that controls
the relative weights of the objective functions.
\end{Description}
%
\begin{Usage}
\begin{verbatim}
MSSearch(msopt, alpha, ...)
\end{verbatim}
\end{Usage}
%
\begin{Arguments}
\begin{ldescription}
\item[\code{msopt}] A list as returned by the \code{\LinkA{MSOpt}{MSOpt}} function.

\item[\code{alpha}] A vector of weights, whose elements must sum to one.
\code{length(alpha)} must be equal to the number of criteria considered, that
is, it must be equal to the length of the \code{criteria} element of the
\code{msopt} argument.

\item[\code{...}] optional arguments (see below).
\end{ldescription}
\end{Arguments}
%
\begin{Details}\relax
INSERIRE PARTE SULLA NORMALIZZAZIONE.

Additional arguments can be specified as follows:
\begin{itemize}

\item{} \code{'Start', sol}: A string and a matrix, used in pair. They provide
a starting solution (\code{sol}) or initial design to the algorithm. By
default the initial solution is randomly generated following the
\emph{SampleDesign} procedure described in Sambo, Borrotti, Mylona and Gilmour
(2017).

\item{} \code{'Restarts', r }: A string and an integer, used in pair. When
\code{r=1}, the default value, the procedure implemented in \code{MSSearch}
results in a local search algorithm that optimizes the objective function
\eqn{f_W}{} starting from one initial design in the design space. These
parameters allows to restart the algorithm \code{r} times. If no initial design
is passed a different starting solution is generated for each iteration, letting
the probability to find a global minimum be higher. \code{Mssearch} returns
the solution that minimizes \eqn{f_W}{} across all the \code{r} iterations.

\item{} \code{'Normalize', c(CritTR, CritSC)}: A string and a vector, used in
pair. By specifying the \code{CritTR} and \code{CritSC} vectors, the user can
establish the normalization factors to be applied to each objective function
before evaluating \eqn{f_W}{}. \code{CritTR} and \code{CritSC} are vectors of
length equal to the number of criteria, whose default elements are 0 and 1
respectively.

\end{itemize}

\end{Details}
%
\begin{Value}
\code{MSSearch} returns a list, whose elements are:
\begin{itemize}

\item{} \code{optsol}: A design matrix. The best solution found.
\item{} \code{optscore}: A vector containing the criteria scores for
\code{optsol}.
\item{} \code{feval}: An integer representing the number of score function
evaluations (number of \eqn{f_W}{} evaluations over all iterations).
\item{} \code{trend}: A vector of length \code{r}. The \eqn{i}{}-th element is
the value that minimizes \eqn{f_W}{} for the \eqn{i}{}-th iteration.

\end{itemize}

\end{Value}
%
\begin{References}\relax
M. Borrotti and F. Sambo and K. Mylona and S. Gilmour. A multi-objective
coordinate-exchange two-phase local search algorithm for multi-stratum
experiments. Statistics \& Computing, 2017.
\end{References}
\inputencoding{utf8}
\HeaderA{Score}{Score}{Score}
%
\begin{Description}\relax
The \code{Score} function returns the optimization criteria values for a
given \code{\LinkA{MSOpt}{MSOpt}} list and design matrix.
\end{Description}
%
\begin{Usage}
\begin{verbatim}
Score(msopt, settings)
\end{verbatim}
\end{Usage}
%
\begin{Arguments}
\begin{ldescription}
\item[\code{msopt}] A list as returned by the function \LinkA{MSOpt}{MSOpt}.

\item[\code{settings}] The design matrix for which criteria scores have to be calculated.
\end{ldescription}
\end{Arguments}
%
\begin{Value}
The vector of scores.
\end{Value}
\inputencoding{utf8}
\HeaderA{multiDoE-package}{multiDoE: What the Package Does (Title Case)}{multiDoE.Rdash.package}
\aliasA{multiDoE}{multiDoE-package}{multiDoE}
%
\begin{Description}\relax
The R package \code{MultiDoE} can be used to construct
multi-stratum experimental designs that optimize up to six statistical
criteria simultaneously. The algorithms implemented in the package to solve
such optimization problems are the \emph{MS-Opt} and \emph{MS-TPLS} algorithms
proposed in Sambo, Borrotti, Mylona, and Gilmour (2017). The former relies on
a search procedure to find locally optimal solutions; the latter, by embedding
MS-Opt in a TPLS framework, is able to generate a good Pareto front
approximation for the optimization problem under study.
Although the implemented methods are designed to handle designs with
experimental factors in at least two different strata, their flexibility
allows their application to even the simplest cases of completely randomized
and randomized block designs. Whatever the case of interest, the designs
manageable by the package are balanced. The user can choose the structure of
the experimental design by defining:
\begin{itemize}

\item{} the number of strata of the experiment;
\item{} the number of experimental factors in each stratum;
\item{} the number of experimental units in each stratum and thus the total number of runs;
\item{} the number of levels of each experimental factor;
\item{} the presence or not of blocking factors.

\end{itemize}

It is possible to choose the a priori model among: the model with main effects
only, the model with main and interaction effects and the full quadratic model.
Finally, estimation of the ratios of error variances in consecutive strata is
required. With the package \code{MultiDoE} it is possible to obtain experimental
designs that optimize any combination of the following criteria: "I", "Id",
"D", "Ds", "A" and "As". Depending on the function used, it is possible to obtain
either a single optimal solution for the optimization problem of interest or
the set of solutions belonging to the (approximate) Pareto front. Once the
Pareto front is available, the package offers: a method for the graphical
visualization of the Pareto front (up to a three-dimensional criteria space);
an objective method for selecting the best design with respect to the entire
set of criteria considered and an objective method for selecting the best design
with respect to to the criteria considered individually.
\end{Description}
%
\begin{Details}\relax
XXX TODO FIX ME XXXX The only function you're likely to need from roxygen2 is [roxygenize()].
Otherwise refer to the vignettes to see how to format the documentation.
This comes from the R/multiDOE-package.R
\end{Details}
%
\begin{References}\relax
M. Borrotti and F. Sambo and K. Mylona and S. Gilmour. A multi-objective
coordinate-exchange two-phase local search algorithm for multi-stratum
experiments. Statistics \& Computing, 2017.
\end{References}
\inputencoding{utf8}
\HeaderA{optMultiCrit}{optMultiCrit}{optMultiCrit}
%
\begin{Description}\relax
The \code{optMultiCrit} function suggests an objective criterion
for the selection of the best experimental design among all Pareto front solutions.
The selection is based on minimizing the euclidean distance in the criteria space
between all the Pareto front designs and an approximate utopian point. By default
the utopian point coordinates are the minimum value reached by every criteria
during an optimization procedure (\code{\LinkA{runTPLS}{runTPLS}}); otherwise
it can be set to a specific value by the user.
\end{Description}
%
\begin{Usage}
\begin{verbatim}
optMultiCrit(ar, ...)
\end{verbatim}
\end{Usage}
%
\begin{Arguments}
\begin{ldescription}
\item[\code{ar}] A list as the \code{megaAR} returned by the \code{runTPLS} function.

\item[\code{...}] optional argument (see below).
\end{ldescription}
\end{Arguments}
%
\begin{Details}\relax
Additional arguments can be specified as follows:
\begin{itemize}

\item{} \code{myUtopianPoint}: A vector containing the utopian point coordinates.

\end{itemize}

\end{Details}
%
\begin{Value}
The \code{optMultiCrit} function returns a list whose elements are:
\begin{itemize}

\item{} \code{solution}: The selected optimal design matrix.
\item{} \code{score}: A vector containing the criteria scores for \code{solution}.

\end{itemize}

\end{Value}
\inputencoding{utf8}
\HeaderA{optSingleCrit}{optSingleCrit}{optSingleCrit}
%
\begin{Description}\relax
The \code{optSingleCrit} function selects the Pareto front designs
that optimizes the individually considered criteria.
\end{Description}
%
\begin{Usage}
\begin{verbatim}
optSingleCrit(ar)
\end{verbatim}
\end{Usage}
%
\begin{Arguments}
\begin{ldescription}
\item[\code{ar}] A list as the \code{megaAR} returned by the \code{runTPLS} function.
\end{ldescription}
\end{Arguments}
%
\begin{Value}
A list whose \eqn{i}{}-th element corresponds to the solution that optimizes
the \eqn{i}{}-th criterion in \code{criteria}. The solution is a list of two elements:
\begin{itemize}

\item{} \code{score}: A vector containing the scores for every element in \code{criteria}.
\item{} \code{solution}: The design matrix.

\end{itemize}

\end{Value}
\inputencoding{utf8}
\HeaderA{plotPareto}{plotPareto}{plotPareto}
%
\begin{Description}\relax
The \code{plotPareto} function returns a graphical representation
(at most 3D) of the Pareto front.
\end{Description}
%
\begin{Usage}
\begin{verbatim}
plotPareto(ar, x, y, z = NULL, mode = T)
\end{verbatim}
\end{Usage}
%
\begin{Arguments}
\begin{ldescription}
\item[\code{ar}] A list as the \code{megaAR} returned by the \code{runTPLS} function.

\item[\code{x}] The criterion on the x axis. It can be one of the following: \code{"I",
"Id", "D", "Ds", "A"} and \code{"As"}.

\item[\code{y}] The criterion on the y axis. It can be one of the following: \code{"I",
"Id", "D", "Ds", "A"} and \code{"As"}.

\item[\code{z}] The criterion on the z axis. It can be one of the following: \code{"I",
"Id", "D", "Ds", "A"} and \code{"As"}.

\item[\code{mode}] When \code{mode=True} the function returns a 3D interactive
chart. When \code{mode=False} it returns a 2D chart in which the \code{z} criteria
values are represented by a color scale.
\end{ldescription}
\end{Arguments}
%
\begin{Value}
The Pareto front chart.
\end{Value}
\inputencoding{utf8}
\HeaderA{runTPLS}{runTPLS}{runTPLS}
%
\begin{Description}\relax
This function implements the \emph{Multi-Stratum Two-Phase Local
Search} (MS-TPLS) algorithm described in Borrotti, Sambo, Mylona and Gilmour
(2017). The MS-TPLS algorithm is useful to obtain exact optimal multi-stratum
designs using a multi-criteria approach. The number of iterations of the MS-TPLS
algorithm must be set by the user. The resulting experimental designs can minimize up
to six criteria simultaneously from the following: I, D, A, Id, Ds and As. The
\code{runTPLS} function is able to provide the set of solutions that build the
approximate Pareto front for the specified optimization problem.
\end{Description}
%
\begin{Usage}
\begin{verbatim}
runTPLS(facts, units, criteria, model, iters, ...)
\end{verbatim}
\end{Usage}
%
\begin{Arguments}
\begin{ldescription}
\item[\code{facts}] A list of vectors representing the distribution of factors
across strata. Each item in the list represents a stratum and the first item
is the highest stratum of the multi-stratum structure of the experiment.
Within the vectors, experimental factors are indicated by progressive integer
from 1 (the first factor of the highest stratum) to the total number of
experimental factors (the last factor of the lowest stratum). Blocking
factors are differently denoted by empty vectors.

\item[\code{units}] A list whose \eqn{i}{}-th element, \eqn{n_i}{}, is the number of
experimental units within each unit at the previous stratum (\eqn{i-1}{}). The
first item in the list, \eqn{n_1}{}, represents the number of experimental
units in the stratum \eqn{0}{}. The latter is defined as the entire experiment,
such that \eqn{n_0 = 1}{}.

\item[\code{criteria}] A list specifying the criteria to be optimized. It can
contain any combination of:
\begin{itemize}

\item{} ``I" : I-optimality
\item{} ``Id" : Id-optimality
\item{} ``D" : D-optimality
\item{} ``A" : Ds-optimality
\item{} ``Ds" : A-optimality
\item{} ``As" : As-optimality

\end{itemize}

These criteria are well explained in Borrotti, Sambo, Mylona and Gilmour (2017).
More detailed information on the available criteria is also given in
\code{\LinkA{MSOpt}{MSOpt}.}

\item[\code{model}] A string which indicates the type of model, among ``main",
``interaction" and ``quadratic".

\item[\code{iters}] An integer indicating the number of iterations of the MS-TPLS
algorithm.

\item[\code{...}] optional arguments (see below).
\end{ldescription}
\end{Arguments}
%
\begin{Details}\relax
Additional arguments can be specified as follows:
\begin{itemize}

\item{} \code{'Restarts', restarts}: A string and an integer, used in pair. \code{r}
defines the number of times the MS-Opt procedure is altogether called within
each iteration of the MS-TPLS algorithm. The default value is \code{r=100}.

\item{} \code{'Levels', levels}: A string and a vector, used in pair. \code{levels}
is a vector containing the number of available levels for each experimental
factor in the argument \code{facts} (blocking factors are excluded). If all
experimental factors share the number of levels one integer is sufficient.

\item{} \code{'Etas', etas}: A string and a list, used in pair. In \code{etas}
the user must specify the ratios of error variance between subsequent strata,
starting from the highest strata. It follows that \code{length(etas)} must be
equal to \code{length(facts)-1}.

\item{} \code{'RestInit', restInit}: A string and an integer, used in pair. Through
these parameters, it is possible to determine how many of the \code{r} iterations
of MS-Opt should be used for each criterion in the first step of the MS-TPLS
algorithm (lines 3-6 of the pseudo-code of MS-TPLS, see Borrotti, Sambo, Mylona
and Gilmour (2017)). The default value is \code{restInit=50}. Let \eqn{n}{} be
the number of criteria under consideration. One can calculate accordingly as
\eqn{r - (n * restInit)}{} the number of times MS-Opt is called in the
second step (lines 7-11 of the pseudo-code of MS-TPLS) of each iteration of MS-TPLS.

\end{itemize}

\end{Details}
%
\begin{Value}
\code{runTPLS} returns a list, whose elements are:
\begin{itemize}

\item{} \code{ar}: A list of length equal to \code{iters}. The \eqn{i}{}-th element
is a list whose elements are:
\begin{itemize}

\item{} \code{nsols}: Number of designs produced during the \eqn{i}{}-th iteration.
\item{} \code{dim}: The criteria space dimension.
\item{} \code{scores}: A matrix of \code{nsols} rows and \code{dim} columns.
Every row contains the value of the criteria for each solution of the
\eqn{i}{}-th iteration.
\item{} \code{solutions}: A list of length equal to \code{nsols} containing the
design matrices produced during the \eqn{i}{}-th iteration. The values of the
criteria corresponding at the first element of \code{solutions} are placed
in the first row of the \code{scores} matrix and so on.

\end{itemize}

\item{} \code{stats}: A list of length equal to \code{iters}. Every element is a
vector of size \eqn{r - (n * restInit) + 1}{}, where \eqn{n}{} is the number of the
considered criteria. The first element represents the number of function
evaluations during the first step of the MS-TPLS algorithm; the \eqn{i}{}-th
element (excluding the first one) is the sum of the number of evaluations for
the \eqn{i}{}-th scalarization and the maximum value in the \code{stats}.

\item{} \code{megaAR}: A list whose elements are:
\begin{itemize}

\item{} \code{nsols}: The number of the Pareto front solutions.
\item{} \code{dim}: The criteria space dimension.
\item{} \code{scores}: A matrix of \code{nsols} rows and \code{dim} columns.
Every row contains the criteria values for each Pareto front design.
\item{} \code{solutions}: A list of length equal to \code{nsols} containing
the design matrices for the Pareto front designs. The values of the criteria
corresponding at the first element of \code{solutions} are placed in the
first row of the \code{scores} matrix and so on.

\end{itemize}


\end{itemize}

\end{Value}
%
\begin{References}\relax
M. Borrotti and F. Sambo and K. Mylona and S. Gilmour. A multi-objective
coordinate-exchange two-phase local search algorithm for multi-stratum
experiments. Statistics \& Computing, 2017.
\end{References}
\printindex{}
\end{document}
